%%%%%%%%%%%%%%%%%%%%%%%%%%%%%%%%%%%%%%%%%
% Awesome Cover Letter
% XeLaTeX Template
% Version 1.1 (9/1/2016)
%
% This template has been downloaded from:
% http://www.LaTeXTemplates.com
%
% Original authors:
% Claud D. Park (posquit0.bj@gmail.com)
% Lars Richter (mail@ayeks.de)
% With modifications by:
% Vel (vel@latextemplates.com)
%
% License:
% CC BY-NC-SA 3.0 (http://creativecommons.org/licenses/by-nc-sa/3.0/)
%
% Important note:
% This template must be compiled with XeLaTeX, the below lines will ensure this
%!TEX TS-program = xelatex
%!TEX encoding = UTF-8 Unicode
%
%%%%%%%%%%%%%%%%%%%%%%%%%%%%%%%%%%%%%%%%%

%----------------------------------------------------------------------------------------
%	PACKAGES AND OTHER DOCUMENT CONFIGURATIONS
%----------------------------------------------------------------------------------------

\documentclass[11pt, a4paper]{awesome-cv} % A4 paper size by default, use 'letterpaper' for US letter

\geometry{left=1in, top=1in, right=1in, bottom=2cm, footskip=.5cm} % Configure page margins with geometry

\fontdir[fonts/] % Specify the location of the included fonts

% Color for highlights
\colorlet{awesome}{awesome-skyblue} % Default colors include: awesome-emerald, awesome-skyblue, awesome-red, awesome-pink, awesome-orange, awesome-nephritis, awesome-concrete, awesome-darknight
%\definecolor{awesome}{HTML}{CA63A8} % Uncomment if you would like to specify your own color

% Colors for text - uncomment and modify
%\definecolor{darktext}{HTML}{414141}
%\definecolor{text}{HTML}{414141}
%\definecolor{graytext}{HTML}{414141}
%\definecolor{lighttext}{HTML}{414141}

\renewcommand{\acvHeaderSocialSep}{\quad\textbar\quad} % If you would like to change the social information separator from a pipe (|) to something else

%----------------------------------------------------------------------------------------
%	PERSONAL INFORMATION
%	Comment any of the lines below if they are not required
%----------------------------------------------------------------------------------------

\photo[circle,noedge,left]{Photo_lab}
\name{Facebook Fellowship}{Yida Wang}
\address{Lillweg 13, Munich, Germany, Technical Unicersity of Munich \\Postal number: 80939}
\mobile{(+49) 151-4368-1357}

\email{yidawang.cn@gmail.com}
\homepage{wangyida.github.io}
\github{wangyida}
\linkedin{Yida Wang}
\skype{Yida Wang}
\stackoverflow{6844155}{yida-wang}
%\twitter{@twit}

\position{Ph.D student{\enskip\cdotp\enskip}Machine Learning and Computer Vision} % Your expertise/fields
%\quote{``Proudly Enrolled in PRIS Lab of BUPT."} % A quote or statement

\makecvfooter{\today}{Yida Wang~~~·~~~Résumé}{\thepage} % Specify the letter footer with 3 arguments: (<left>, <center>, <right>), leave any of these blank if they are not needed

%----------------------------------------------------------------------------------------
%	RECIPIENT/POSITION/LETTER INFORMATION
%	All of the below lines must be filled out
%----------------------------------------------------------------------------------------

\recipient{Facebook}{Initech Inc.\\4120 Freidrich Ln.\\Austin, TX 78744} % The company being applied to

\letterdate{\today} % The date on the letter, default is the date of compilation

\lettertitle{Ph.D Fellowship Application for Facebook} % The title of the letter

\letteropening{Dear Mr./Ms./Dr. LastName,} % How the letter is opened

\letterclosing{Sincerely,} % How the letter is closed

\letterenclosure[Attached]{Curriculum Vitae} % Any enclosures with the letter

\makecvfooter{\today}{Yida Wang~~~Research Summary~~~Facebook Ph.D Fellowship}{\thepage} % Specify the letter footer with 3 arguments: (<left>, <center>, <right>), leave any of these blank if they are not needed

%----------------------------------------------------------------------------------------

\begin{document}

\makecvheader % Print the header

\lettersection{Topic}

Deep learning techniques for representation and processing of high dimensional data for complex visual and linguistic tasks.
\begin{itemize}
	\itemsep0em
	\item CommAI
	\item Computer Vision
	\item Machine Learning
\end{itemize}

\lettersection{About Me}

I am Yida Wang, Ph.D candidate in Technical University of Munich, involved in Chair for computer aided medical image processing and augmented reality, Department of Information and mathematics. 
Technical University of Munich is the 1st ranking University in Germany currently with a 9th worldwide position in computer science which is just my current Ph.D major. 
I received B.E. and M.E. degrees from Beijing University of Posts and Telecommunications, Beijing, China in 2014 and 2017, respectively for the 1st ranking major in China for Information and Telecommunication engineering. 
My research interests include information theories, pattern recognition and computer vision. 
I have published several technical papers in international conferences with some oral presentations. I was invited by Microsoft Research for attending Microsoft Faculty Summit in 2016 for project of "CNTK on Mac: 2D Object Restoration and Recognition Based on 3D Model" which was awarded the second prize for Microsoft Open Source Challenge. I was also sponsored by Google Summer of Code project twice for deep learning projects together with OpenCV organization. He was named of Excellent Graduate Student of Beijing City twice in 2014 and 2017 and been awarded for National Scholarship for Graduate Students in 2016.
My ability and experience on machine learning and computer vision are solid for my future research and also willing to join research teams in Facebook for novel algorithms and applications. 

\lettersection{Area of Focus}

I have experience and passion on computer vision and machine learning research, my Ph.D topic is development of deep learning techniques for representation and processing of 3d data for complex visual tasks. 
For CVPR submission this year, I have got 2 state of the art results for advanced sparse metric learning methods both for 3D and 2D object partial semantic segmentation tasks based on probabilistic model designed based on Bayesian inference. 
Such method could even be adopted in stock prediction which is already a good solution in a startup in China.
I have contributed several generative models which could be used for image, point clouds and video recognition. 
I have published several papers in good quality in popular conferences and journals such as ICIP, CCBR, ACCV, CVPR (in submission) and IEEE TIP with the guidance of my master adviser Dr. Weihong Deng from Pattern Recognition and Intelligent System Laboratory of Beijing University of Posts and Telecommunications and Ph.D advisor Dr. Federico Tombari from Technical University of Munich. 
I am the only one in my previous group last year ever accepted the highest honor for the National Scholarship for master students of China for my research of 2016 while my other academic achievements are listed in my Resume. 
I value contributions from great people much, so I always use open source tools to develop novel methods and useful codes for my own researches and the whole research community. 

For projects experiences, I’m invited by Dr. Judith Bishop from Microsoft Research Redmond for a presentation on Microsoft Faculty Summit 2016 because I received the global 2nd prize in deep learning challenge. I am also one of a project member of deep learning toolbox: tiny-dnn with the help of my mentor Dr. Stefano Fabri and Manuele Tamburrano from Consorzio Roma Ricerche because I had internship with OpenCV for 2 summers from 2015 to 2016 in Computer Vision and also been involving in practical machine learning with sponsorship from Google within these 2 years. So my experience is not just limited to computer vision related topics, but also other techniques or algorithms such as 3D models utilization techniques and large scale text data analysis.

As an athlete of University track and field team for 7 years, I won the medals both in national and interstate competition more than once. 

\lettersection{On Going Research}

I am doing a novel topic called "Nebula Variational Coder with Sparse Metric Learning for Supervised and Unsupervised Prediction".
  I propose a variational inference model trained with sparse metric learning which can be applied on different deep architectures for specific tasks including classification and segmentation. 
  Our sparse metric learning is proposed for complicated metric learning with stable time complexity for the first time which is far faster than traditional metric learning such as triplet training. 
  It performs well for data with nearly all kinds of dimensions such as audio, images and 3D point clouds.
  Traditional variational inference for vision tasks still has two problems which is over-fitting for encoder trained with KL divergence with posterior and suffering from local optima problem.
  Our model enhances the semantic learning capability for latent variables based on Gaussian assumption in variational inference for parameter optimization by utilizing sparse metric learning with additional information. 
  Relationships of samples in mini batch are represented with tensor of different orders in Cartesian coordinate system and calculated with binary tensor calculus. 
  Nebula anchors are also introduced as training assistants for latent variables which can balance unsupervised learning and supervised learning for training and keep our model easy to applied for testing. 
  With the help of nebula anchors and sparse learning, latent variables of our variational coder form hyper clusters which is both useful for semantic coding and fitting for Gaussian assumption.

\lettersection{Applicability to Facebook}

As I have contributed to Google Open Source, OpenCV and Microsoft Open Source community for more than 3 years on computer vision and machine learning libraries, I am using Caffe2 to modify PointNet into C++ library from TensorFlow. The performance on 3D point clouds segmentation and prediction will be greatly improved based on Caffe2 framework.

I am willing to contribute to Facebook Open Source community based on computer vision and machine learning libraries like Caffe2.

I also plan to publish top quality publications such as ICCV, CVPR, TIP and TPAMI in colaboration with Facebook research and maintain frontier open source machine learning libraries. Both of my supervisor and myself are interested in joint work from TUM and Facebook, especially on publications with top quality and real time applications. We have many good publications and real products every year because both of me and my advisor are experienced open source contributors and mentors.

I can be a fully involved researcher with Facebook research and both of me and my advisor are willing to have some amazing joint projects and publication.
%----------------------------------------------------------------------------------------

\end{document}
