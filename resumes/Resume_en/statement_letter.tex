%%%%%%%%%%%%%%%%%%%%%%%%%%%%%%%%%%%%%%%%%
% Awesome Cover Letter
% XeLaTeX Template
% Version 1.1 (9/1/2016)
%
% This template has been downloaded from:
% http://www.LaTeXTemplates.com
%
% Original authors:
% Claud D. Park (posquit0.bj@gmail.com)
% Lars Richter (mail@ayeks.de)
% With modifications by:
% Vel (vel@latextemplates.com)
%
% License:
% CC BY-NC-SA 3.0 (http://creativecommons.org/licenses/by-nc-sa/3.0/)
%
% Important note:
% This template must be compiled with XeLaTeX, the below lines will ensure this
%!TEX TS-program = xelatex
%!TEX encoding = UTF-8 Unicode
%
%%%%%%%%%%%%%%%%%%%%%%%%%%%%%%%%%%%%%%%%%

%----------------------------------------------------------------------------------------
%	PACKAGES AND OTHER DOCUMENT CONFIGURATIONS
%----------------------------------------------------------------------------------------

\documentclass[11pt, a4paper]{awesome-cv} % A4 paper size by default, use 'letterpaper' for US letter

\geometry{left=1in, top=1in, right=1in, bottom=2cm, footskip=.5cm} % Configure page margins with geometry

\fontdir[fonts/] % Specify the location of the included fonts

% Color for highlights
\colorlet{awesome}{awesome-skyblue} % Default colors include: awesome-emerald, awesome-skyblue, awesome-red, awesome-pink, awesome-orange, awesome-nephritis, awesome-concrete, awesome-darknight
%\definecolor{awesome}{HTML}{CA63A8} % Uncomment if you would like to specify your own color

% Colors for text - uncomment and modify
%\definecolor{darktext}{HTML}{414141}
%\definecolor{text}{HTML}{414141}
%\definecolor{graytext}{HTML}{414141}
%\definecolor{lighttext}{HTML}{414141}

\renewcommand{\acvHeaderSocialSep}{\quad\textbar\quad} % If you would like to change the social information separator from a pipe (|) to something else

%----------------------------------------------------------------------------------------
%	PERSONAL INFORMATION
%	Comment any of the lines below if they are not required
%----------------------------------------------------------------------------------------

\photo[circle,noedge,left]{Photo_lab}
\name{Statement}{Yida Wang}
\address{Lillweg 13, Munich, Germany, Technical Unicersity of Munich \\Postal number: 80939}
\mobile{(+49) 151-2710-5028}

\email{yidawang.cn@gmail.com}
\homepage{wangyida.github.io}
\github{wangyida}
\linkedin{Yida Wang}
\skype{Yida Wang}
\stackoverflow{6844155}{yida-wang}
%\twitter{@twit}

\position{Ph.D student{\enskip\cdotp\enskip}Machine Learning and Computer Vision} % Your expertise/fields
%\quote{``Proudly Enrolled in PRIS Lab of BUPT."} % A quote or statement

\makecvfooter{\today}{Yida Wang~~·~~Résumé}{\thepage} % Specify the letter footer with 3 arguments: (<left>, <center>, <right>), leave any of these blank if they are not needed

%----------------------------------------------------------------------------------------
%	RECIPIENT/POSITION/LETTER INFORMATION
%	All of the below lines must be filled out
%----------------------------------------------------------------------------------------

\recipient{Committee}{Initech Inc.\\4120 Freidrich Ln.\\Austin, TX 78744} % The company being applied to

\letterdate{\today} % The date on the letter, default is the date of compilation

\lettertitle{Statement of Me} % The title of the letter

\letteropening{Dear Mr./Ms./Dr. LastName,} % How the letter is opened

\letterclosing{Sincerely,} % How the letter is closed

\letterenclosure[Attached]{Curriculum Vitae} % Any enclosures with the letter

\makecvfooter{\today}{Yida Wang~~~Research Summary~~~}{\thepage} % Specify the letter footer with 3 arguments: (<left>, <center>, <right>), leave any of these blank if they are not needed

%----------------------------------------------------------------------------------------

\begin{document}

\makecvheader % Print the header

\lettersection{About Me}

% Me
I am Yida Wang, Ph.D candidate in Technical University of Munich. I work in Chair for computer aided medical image processing and augmented reality, Department of Information and mathematics. 
I received B.E and M.E degrees from Beijing University of Posts and Telecommunications, China.
% Research
My research interests include information theories, pattern recognition and computer vision. 
Several publications in popular conferences and journals such as IEEE TIP, IEEE ICIP, IEEE 3DV and ACCV are made with the guidance of my master adviser Dr. Weihong Deng from Pattern Recognition and Intelligent System Laboratory of Beijing University of Posts and Telecommunications and Ph.D advisor Dr. Federico Tombari from Technical University of Munich. 
% Practical experience
For practical experiences, I was ever invited by Microsoft Research for attending Microsoft Faculty Summit for an award of the second prize for Microsoft Open Source Challenge. I was also sponsored by Google Summer of Code project twice for deep learning projects supervised with OpenCV. 
% Honors
Furthermore, I was named of Excellent Graduate Student of Beijing City for both bachelor and master degree. Specifically, I am honored with the National Scholarship for master students which is top award in China for my research. 
My ability and experience on machine learning and computer vision are solid for my future research and I am also willing to participate in advanced research projects for machine learning and computer vision. 

As an athlete of University track and field team for 7 years, I won the medals both in national and interstate competition more than once. 

\lettersection{Area of Focus}

I apply deep learning techniques for representation and processing of high dimensional data for complex visual and linguistic tasks including general artificial intelligence, computer vision and machine learning.
I have experience and passion on computer vision and machine learning research, my Ph.D topic is development of deep learning techniques designed for 3d data for complex visual tasks. 
Apart from that, I also do research on 2D and 3D data semantic translation with visual techniques.
Tensor algebra and statistics are two major backbones supporting my research on real senarios. Some of my work in generative inference models are strongly supported by statistical theories. My research for deep metric learning are also based on tensor algebra in high dimensionality.


\end{document}
